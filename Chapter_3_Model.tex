%=============================================================================================
%                       Chapter 3: Physical Model and Theoretical Formulation
%=============================================================================================

\chapter{Raman-Driven Quantum Beat Laser: Model and Dynamics}
\label{chp3}
\newpage

\section{Introduction}

This chapter presents the detailed theoretical formulation of the Raman-Driven Quantum Beat Laser (RDQBL) system. We develop the complete quantum mechanical model, derive the master equation governing system evolution, and establish the mathematical framework for analyzing quantum steering in this active quantum system.

The chapter is organized as follows: first, we describe the physical system configuration and atomic level structure; next, we present the full Hamiltonian in the interaction picture; then, we derive the density matrix equations of motion including cavity losses and atomic decay; finally, we develop the covariance matrix formalism for quantifying quantum steering in the output field.

\section{Physical System Configuration}

\subsection{Four-Level Atomic System}

The RDQBL employs a four-level atomic system in cascade configuration with energy levels:
\begin{itemize}
	\item $|a\rangle$: Ground state (stable, lifetime $\rightarrow \infty$)
	\item $|b\rangle$: Intermediate state (stable, lifetime $\rightarrow \infty$)
	\item $|c\rangle$: Excited state (short-lived, decays rapidly)
\end{itemize}

The relevant dipole transitions are:
\begin{itemize}
	\item $|a\rangle \leftrightarrow |b\rangle$: Resonant with cavity mode 1 (frequency $\nu_1$)
	\item $|b\rangle \leftrightarrow |c\rangle$: Resonant with cavity mode 2 (frequency $\nu_2$)
	\item $|a\rangle \leftrightarrow |c\rangle$: Driven by external classical field (Raman process)
\end{itemize}

\subsection{Schematic Representation}

[Insert Figure 3.1: RDQBL System Schematic showing four-level configuration, cavity modes, and driving fields]

\section{The Hamiltonian}

\subsection{Full Hamiltonian in Interaction Picture}

In the electric dipole and rotating wave approximations, the interaction picture Hamiltonian is:
\begin{equation}
H = \hbar g_1 a_1 |a\rangle\langle b| + \hbar g_2 a_2 |b\rangle\langle c| - \frac{\hbar\Omega}{2}e^{-i\phi}|a\rangle\langle c| + \text{H.c.}
\end{equation}

where:
\begin{itemize}
	\item $a_1, a_2$: Annihilation operators for cavity modes 1 and 2
	\item $g_1, g_2$: Cavity-atom coupling strengths
	\item $\Omega e^{-i\phi}$: Rabi frequency and phase of external driving field
	\item H.c.: Hermitian conjugate
\end{itemize}

\subsection{Parameters and Their Physical Significance}

\textbf{Coupling Constants:} $g_1, g_2$ represent the strength of light-matter interaction. Typical values: $g_i \sim 10^6 - 10^7$ Hz.

\textbf{Rabi Frequency:} $\Omega$ characterizes the strength of the external driving field inducing the $|a\rangle \leftrightarrow |c\rangle$ transition. Controllable experimentally via laser intensity.

\textbf{Relative Phase:} $\phi$ is the relative phase between the two Raman driving fields. This parameter enables control over quantum steering directionality.

\section{Equations of Motion}

\subsection{Density Matrix Evolution}

The time evolution of density matrix elements is governed by the master equation including decay terms:
\begin{equation}
\dot{\rho}_{ij} = -\frac{i}{\hbar}\sum_k(V_{ik}\rho_{kj} - \rho_{ik}V_{kj}) - \frac{1}{2}\sum_k(\Gamma_{ik}\rho_{kj} + \rho_{ik}\Gamma_{kj})
\end{equation}

where $\Gamma_{ik}$ are decay coefficients for atomic transitions.

\subsection{Relevant Density Matrix Elements}

For the cascade system, the relevant coherences are:
\begin{itemize}
	\item $\rho_{ab} = \langle a|\rho|b\rangle$: Coherence between ground and intermediate states
	\item $\rho_{bc} = \langle b|\rho|c\rangle$: Coherence between intermediate and excited states
	\item $\rho_{ac} = \langle a|\rho|c\rangle$: Coherence between ground and excited states (driven by external field)
\end{itemize}

\subsubsection{Equations for Coherences}

\begin{align}
\dot{\rho}_{ab} &= -\gamma\rho_{ab} + \frac{i\Omega}{2}e^{-i\phi}\rho_{cb} - ig_1(a_1\rho_{bb} - \rho_{aa}a_1) + ig_2\rho_{ac}a_2^\dagger \\
\dot{\rho}_{bc} &= -\gamma\rho_{bc} - \frac{i\Omega}{2}e^{-i\phi}\rho_{ba} - ig_2(a_2\rho_{cc} - \rho_{bb}a_2) - ig_1a_1^\dagger\rho_{ac}
\end{align}

where $\gamma$ is the atomic decay rate for the excited state $|c\rangle$.

\section{Cavity Field Dynamics}

\subsection{Master Equation for Reduced Density Operator}

Tracing over atomic degrees of freedom and focusing on the cavity field, the evolution of $\rho_f$ (field reduced density matrix) is:
\begin{equation}
\dot{\rho}_f = \frac{1}{\hbar}[H_{\text{eff}}, \rho_f] + \mathcal{L}_{\text{cavity}}[\rho_f]
\end{equation}

where the cavity loss Liouvillian is:
\begin{equation}
\mathcal{L}_{\text{cavity}}[\rho_f] = \kappa_1(a_1\rho_f a_1^\dagger - \frac{1}{2}\{a_1^\dagger a_1, \rho_f\}) + \kappa_2(a_2\rho_f a_2^\dagger - \frac{1}{2}\{a_2^\dagger a_2, \rho_f\})
\end{equation}

with cavity decay rates $\kappa_1, \kappa_2$.

\section{Covariance Matrix Formulation}

\subsection{Quadrature Operators}

The quadrature operators for the two cavity modes are defined as:
\begin{align}
X_1 &= \frac{1}{\sqrt{2}}(a_1 + a_1^\dagger), \quad P_1 = \frac{i}{\sqrt{2}}(a_1^\dagger - a_1) \\
X_2 &= \frac{1}{\sqrt{2}}(a_2 + a_2^\dagger), \quad P_2 = \frac{i}{\sqrt{2}}(a_2^\dagger - a_2)
\end{align}

\subsection{Two-Mode Covariance Matrix}

For a two-mode Gaussian state, all correlations are captured by the covariance matrix:
\begin{equation}
V = \begin{pmatrix}
V_1 & C_{12} \\
C_{12}^T & V_2
\end{pmatrix}
\end{equation}

where:
\begin{align}
V_1 &= \begin{pmatrix}
\langle X_1^2 \rangle - \langle X_1 \rangle^2 & \frac{1}{2}(\langle X_1 P_1 + P_1 X_1 \rangle - 2\langle X_1\rangle\langle P_1 \rangle) \\
\frac{1}{2}(\langle X_1 P_1 + P_1 X_1 \rangle - 2\langle X_1\rangle\langle P_1 \rangle) & \langle P_1^2 \rangle - \langle P_1 \rangle^2
\end{pmatrix} \\
C_{12} &= \begin{pmatrix}
\langle X_1 X_2 \rangle - \langle X_1\rangle\langle X_2 \rangle & \frac{1}{2}(\langle X_1 P_2 + P_2 X_1 \rangle - 2\langle X_1\rangle\langle P_2 \rangle) \\
\frac{1}{2}(\langle X_2 P_1 + P_1 X_2 \rangle - 2\langle X_2\rangle\langle P_1 \rangle) & \langle P_1 P_2 \rangle - \langle P_1\rangle\langle P_2 \rangle
\end{pmatrix}
\end{align}

\section{Quantifying Quantum Steering}

\subsection{Steering Criterion for Gaussian States}

For the two-mode Gaussian state with covariance matrix $V$, steering from subsystem 1 to subsystem 2 is quantified as:
\begin{equation}
S^{1 \to 2} = \max\left\{0, \frac{1}{2}\log_2\frac{\det V_1}{4\det V_{\text{out}}^T}\right\}
\end{equation}

where $V_{\text{out}}^T$ is evaluated from the conditional state after projective measurements by party 1.

\subsection{Symplectic Eigenvalues}

For Gaussian states, the steering criterion can equivalently be expressed using symplectic eigenvalues of the partial transpose of the covariance matrix:
\begin{equation}
S^{1 \to 2} > 0 \iff \tilde{\nu}_- < \frac{1}{2}
\end{equation}

where $\tilde{\nu}_-$ is the smallest symplectic eigenvalue.

\subsection{Comparison with Entanglement}

Logarithmic negativity (entanglement):
\begin{equation}
E = \max\{0, -\log_2(2\tilde{\nu}_-)\}
\end{equation}

Key difference: Steering criterion depends on determinant of subsystem 1 covariance ($\det V_1$), while entanglement criterion depends on determinants of both subsystems and their cross-correlations.

\section{Initial Conditions and System Parameters}

\subsection{Input Cavity Field State}

We consider the initial cavity field as a product of two independent single-mode Gaussian states:
\begin{equation}
\rho_{\text{in}} = \rho_1 \otimes \rho_2
\end{equation}

Each single-mode state is characterized by:
\begin{itemize}
	\item Non-classicality parameter: $\tau$ (ranging 0 to 1/2 for squeezed states)
	\item Purity: $\mu$ (ranging 0 for maximally mixed to 1 for pure)
\end{itemize}

\subsection{Typical Parameters}

Realistic RDQBL parameters for simulations:
\begin{itemize}
	\item Coupling strengths: $g_1 = g_2 = 16$ kHz
	\item External field Rabi frequency: $\Omega = 2160$-$3160$ kHz (tunable)
	\item Relative phase: $\phi \in [0, 2\pi]$ (tunable)
	\item Atomic decay rate: $\gamma = 20$ kHz
	\item Cavity decay rates: $\kappa_1, \kappa_2 = 0$-$0.1$ kHz
	\item Atom injection rate: $r_a = 10$ kHz
\end{itemize}

\section{Chapter Summary}

This chapter has provided:
\begin{enumerate}
	\item Complete description of the RDQBL four-level system configuration
	\item Full quantum Hamiltonian in interaction picture
	\item Master equations for cavity field evolution including losses
	\item Covariance matrix formalism for Gaussian state analysis
	\item Mathematical criteria for steering quantification
	\item Physical interpretation of tunable parameters
	\item Realistic system parameters for numerical studies
\end{enumerate}

These foundations enable the numerical investigations presented in Chapter \ref{chp4}, where we solve the system dynamics and analyze steering generation and control.

