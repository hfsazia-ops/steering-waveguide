%=============================================================================================
%                           Chapter 2: Theory and Literature Review
%=============================================================================================

\chapter{Quantum Correlations, Continuous Variables, and Quantum Steering}
\label{chp2}
\newpage

% ============================= SECTION 2.1: INTRODUCTION =============================
\section{Introduction}

Quantum correlations form the fundamental basis of quantum information science, representing non-classical relationships between subsystems that have no classical analogues. Unlike classical information where correlations arise from shared randomness, quantum correlations emerge from the intrinsic non-commutativity of quantum observables and the superposition principle. This chapter establishes the theoretical foundation for understanding quantum steering in continuous-variable systems, with particular emphasis on the RDQBL platform.

We begin by reviewing the hierarchy of quantum correlations—a central organizational principle that categorizes different types of quantum correlations by their strength and scope. We then shift focus to continuous-variable systems and their advantages over discrete-variable approaches. Finally, we examine quantum steering specifically: its mathematical formulation, quantification criteria for Gaussian states, and its unique asymmetric properties that distinguish it from entanglement and Bell nonlocality.

% ============================= SECTION 2.2: LITERATURE REVIEW =============================
\section{Literature Review: Quantum Correlations in Quantum Information}

\subsection{Overview of Quantum Steering}

Quantum correlations have been at the heart of foundational quantum mechanics debates since Einstein, Podolsky, and Rosen (EPR) challenged the completeness of quantum theory in 1935 \cite{einsteinPodolskyRosen1935}. Their paradox highlighted the apparent non-locality inherent in quantum mechanics—that measurements on one system can instantaneously affect another distant system.

Over the past century, several types of quantum correlations have been identified and rigorously formalized:

\begin{enumerate}
	\item \textbf{Quantum Entanglement:} Two or more subsystems are in a non-separable state such that the overall state cannot be written as a product of individual subsystem states. Entanglement is symmetric—if subsystem A is entangled with B, then B is entangled with A.

	\item \textbf{Bell Nonlocality:} A stronger form of correlation where measurements on one system cannot be explained by any local hidden-variable model. Violates Bell inequalities. Symmetric under party exchange.

	\item \textbf{Quantum Steering:} An asymmetric correlation where one party's measurements can influence (``steer'') the conditional state of another distant party in a manner incompatible with local realism. First formally defined by Schrödinger \cite{schrodingerDiscussion1935} as a response to the EPR paradox.

	\item \textbf{Quantum Discord:} A broader measure of quantum correlations that captures non-classical correlations beyond entanglement. Present in many separable states. Most general measure.
\end{enumerate}

\subsection{Hierarchy of Quantum Correlations: Discord $\supseteq$ Entanglement $\supseteq$ Steering $\supseteq$ Bell Nonlocality}

A fundamental insight in quantum information theory is that different types of quantum correlations form a strict hierarchy \cite{brunnerBellNonlocality2014}:

\begin{equation}
\text{Discord} \supseteq \text{Entanglement} \supseteq \text{Steering} \supseteq \text{Bell Nonlocality}
\end{equation}

where $X \supseteq Y$ means X is a superset of Y (i.e., all states exhibiting Y also exhibit X, but not vice versa).

\textbf{Bell Nonlocality:} The most restrictive class. If a state violates Bell inequalities, it exhibits all forms of correlations below it.

\textbf{Steering:} Intermediate in strength. Asymmetric—one party may steer another without the converse being true. Every steerable state exhibits entanglement, but not every entangled state is steerable.

\textbf{Entanglement:} States that are non-separable. Symmetric under party exchange. Always exhibits discord, but may not exhibit steering or nonlocality.

\textbf{Discord:} Widest class, present in many separable states. Classical correlations can coexist with quantum discord. Some separable states exhibit discord.

This hierarchy has profound implications: focusing specifically on steering (our work) targets the optimal balance between correlation strength and operational utility, particularly for one-sided device-independent protocols where only one party needs trusted devices.

% ============================= SECTION 2.3: ENTANGLEMENT IN CV =============================
\section{Entanglement in Continuous Variables}

\subsection{Advantages of Continuous-Variable Systems}

Continuous-variable (CV) quantum information processing offers significant practical advantages over discrete-variable (DV) systems:

\begin{itemize}
	\item \textbf{Deterministic Generation:} Entangled CV states can be generated deterministically through parametric down-conversion and squeezed light, without requiring post-selection.

	\item \textbf{Optical Compatibility:} CV observables (quadrature amplitudes) are naturally measured via homodyne detection using standard optical components. Easy integration with existing fiber-optic infrastructure.

	\item \textbf{Scalability:} Continuous variables allow more quantum information encoding per physical system compared to qubits.

	\item \textbf{Experimental Simplicity:} Homodyne detection is experimentally simpler than single-photon counting required in DV systems.

	\item \textbf{Gaussian State Framework:} Many CV states and operations can be represented using Gaussian formalism, enabling efficient computation and analysis.
\end{itemize}

\subsection{Quadrature Operators and Gaussian States}

The fundamental observables in CV quantum optics are the \textbf{quadrature operators}:
\begin{equation}
X = \frac{1}{\sqrt{2}}(a + a^\dagger), \quad P = \frac{i}{\sqrt{2}}(a^\dagger - a)
\end{equation}

These satisfy the canonical commutation relation:
\begin{equation}
[X, P] = i
\end{equation}

\textbf{Gaussian States} are quantum states whose Wigner function (quasi-probability distribution) is Gaussian. For bipartite systems, Gaussian states are fully characterized by:
\begin{enumerate}
	\item Mean values (first moments): $\langle X_1 \rangle, \langle P_1 \rangle, \langle X_2 \rangle, \langle P_2 \rangle$
	\item Covariance matrix (second-order moments)
\end{enumerate}

The two-mode covariance matrix is:
\begin{equation}
V = \begin{pmatrix}
V_1 & C_{12} \\
C_{12}^T & V_2
\end{pmatrix}
\end{equation}

where each $V_j$ is a $2 \times 2$ block:
\begin{equation}
V_j = \begin{pmatrix}
\langle X_j^2 \rangle - \langle X_j \rangle^2 & \langle\{X_j, P_j\}\rangle/2 - \langle X_j \rangle\langle P_j \rangle \\
\langle\{X_j, P_j\}\rangle/2 - \langle X_j \rangle\langle P_j \rangle & \langle P_j^2 \rangle - \langle P_j \rangle^2
\end{pmatrix}
\end{equation}

\subsection{Quantifying Entanglement in CV: Logarithmic Negativity}

For Gaussian states, entanglement is quantified by \textbf{logarithmic negativity}:
\begin{equation}
E = \max\{0, -\log_2(2\tilde{\eta}_-)\}
\end{equation}

where $\tilde{\eta}_-$ is the smallest symplectic eigenvalue of the partial transpose of the covariance matrix.

A state is entangled if $E > 0$. The metric $\tilde{\eta}_-$ is computed from:
\begin{equation}
2\tilde{\eta}_{\pm}^2 = \zeta \pm \sqrt{\zeta^2 - 4\det V}
\end{equation}
where $\zeta = \det V_1 + \det V_2 - 2\det C_{12}$.

\subsection{Coherent and Squeezed States}

Important classes of CV states include:

\textbf{Coherent States:} Pure states minimizing uncertainty, $|\alpha\rangle = e^{-|\alpha|^2/2}\sum_{n=0}^{\infty} \frac{\alpha^n}{\sqrt{n!}}|n\rangle$.

\textbf{Squeezed States:} States with reduced variance in one quadrature at the expense of increased variance in the orthogonal quadrature. Squeeze parameter $r$ quantifies the squeezing degree.

These states form the foundation for entanglement generation in parametric processes and laser systems.

% ============================= SECTION 2.4: STEERING IN CONTINUOUS VARIABLES =============================
\section{Quantum Steering in Continuous Variables}

\subsection{Formal Definition and Steering Criteria for Gaussian States}

Quantum steering is formalized as follows: Party A (Alice) can steer Party B (Bob) if Alice's local measurements on her subsystem produce conditional states for Bob that cannot be explained by any local hidden-variable model compatible with realism.

For Gaussian states with covariance matrix $V$, the steering criterion derived from the PPT (Positive Partial Transpose) condition is \cite{kogiastQuantificationGaussianQuantum2015}:
\begin{equation}
S^{A \to B} = \max\left\{0, \frac{1}{2}\log_2\frac{\det V_A}{4\det V_{out}^T}\right\}
\end{equation}

where:
\begin{itemize}
	\item $V_A = \det V_1$ is the covariance determinant for Alice's subsystem
	\item $V_{out}^T$ is evaluated over the conditional states Bob can access
\end{itemize}

If $S^{A \to B} > 0$, steering from A to B is demonstrated.

\subsection{Asymmetry of Steering: The Key Distinction}

The defining characteristic of steering is its \textbf{asymmetry}. Unlike entanglement and Bell nonlocality (which are symmetric), steering can be directional:

\begin{equation}
S^{A \to B} > 0 \text{ while } S^{B \to A} = 0 \text{ (Asymmetric Steering)}
\end{equation}

This directionality has profound practical implications:
\begin{itemize}
	\item In asymmetric steering, only one party needs a trusted measurement device
	\item The steered party (Bob) can be untrusted or even adversarial
	\item Perfect for one-sided device-independent protocols
\end{itemize}

\subsection{Entanglement versus Steering}

While related, steering and entanglement are distinct:

\begin{itemize}
	\item \textbf{Entanglement:} Correlation structure of the state itself, symmetric, quantifies by logarithmic negativity
	\item \textbf{Steering:} Ability to influence conditional states through measurement, asymmetric, directional
\end{itemize}

An entangled state may or may not be steerable in both directions. A steerable state is necessarily entangled, but an entangled state need not be steerable (if both parties have local hidden-variable descriptions).

% ============================= SECTION 2.5: EFFECT OF LOSS AND DECOHERENCE =============================
\section{Effect of Loss and Decoherence on Quantum Correlations}

\subsection{Lindblad Master Equation and Loss Operators}

Real quantum systems are open systems interacting with their environment. The dynamics are governed by the Lindblad master equation:
\begin{equation}
\frac{d\rho}{dt} = -\frac{i}{\hbar}[H, \rho] + \sum_k \left(L_k \rho L_k^\dagger - \frac{1}{2}\{L_k^\dagger L_k, \rho\}\right)
\end{equation}

For a two-mode cavity system with loss rates $\kappa_1, \kappa_2$, the loss Liouvillian is:
\begin{equation}
\mathcal{L}_{\text{loss}}[\rho] = \kappa_1\left(a_1\rho a_1^\dagger - \frac{1}{2}\{a_1^\dagger a_1, \rho\}\right) + \kappa_2\left(a_2\rho a_2^\dagger - \frac{1}{2}\{a_2^\dagger a_2, \rho\}\right)
\end{equation}

\subsection{Robustness of Entanglement versus Steering in Waveguides}

Previous work on coupled lossy waveguides \cite{RaiDasAgarwal2010} demonstrated:

\begin{itemize}
	\item Entanglement shows considerable robustness against material loss
	\item Loss parameters $\gamma/J$ up to 1/10 still preserve entanglement
	\item Logarithmic negativity decays but does not vanish rapidly
	\item Different initial states (photon number, NOON, squeezed) show varying robustness
\end{itemize}

Key finding: \textbf{Waveguide structures are reasonably robust against loss effects and appropriate for quantum circuits}.

This provides motivation for investigating steering robustness in similar systems—if entanglement persists, can steering (a potentially stronger resource) also be preserved?

% ============================= SECTION 2.6: OPTICAL WAVEGUIDES AS QM SYSTEMS =============================
\section{Optical Waveguides as Quantum Systems}

\subsection{Basics of Waveguide Coupling and Evanescent Fields}

Coupled optical waveguides represent a passive quantum platform where two single-mode waveguides interact through evanescent field overlap. The system is governed by coupled-mode theory \cite{laiNonclassicalFieldsLinear1991}.

The quantum Hamiltonian for coupled waveguides:
\begin{equation}
H = \hbar\omega(a^\dagger a + b^\dagger b) + \hbar J(a^\dagger b + b^\dagger a)
\end{equation}

where:
\begin{itemize}
	\item $J$ is the coupling strength (depends on waveguide separation)
	\item $a, b$ are annihilation operators for the two modes
\end{itemize}

\subsection{Losses in Realistic Systems}

Real waveguides experience material losses (absorption, scattering):
\begin{equation}
\gamma \text{ (loss rate)} = \frac{2.3 \times \alpha}{10} \text{ (in natural units from dB/cm)}
\end{equation}

Typical parameters:
\begin{itemize}
	\item Silica: $\gamma \approx 3 \times 10^9$ s$^{-1}$, $\gamma/J \approx 1/50$ (excellent)
	\item LiNbO$_3$: $\gamma \approx 3 \times 10^9$ s$^{-1}$, $\gamma/J \approx 1/7$ (moderate)
	\item AlGaAs: $\gamma \approx 2.7 \times 10^{10}$ s$^{-1}$, $\gamma/J \approx 1/10$ (moderate)
\end{itemize}

\subsection{Relevance for Quantum Circuits and Quantum Photonics}

Coupled waveguides serve as basic building blocks for integrated quantum circuits \cite{PolitiSilicaonSilicon2008}. Success in generating robust entanglement in these passive structures motivates investigation of steering—potentially a more useful resource for practical quantum information protocols.

% ============================= SECTION 2.7: ACTIVE QUANTUM SYSTEMS AND STEERING =============================
\section{Application and Motivation: Why Steering in Lossy Waveguides Matters}

\subsection{Role in Photonic Quantum Networks}

Quantum steering in optical systems enables:

\begin{itemize}
	\item \textbf{One-Sided Device-Independent (1sDI) Quantum Networks:} Asymmetric steering allows distributed quantum networks where only central nodes need trusted devices \cite{branciardOnesidedDeviceIndependentQuantum2012}.

	\item \textbf{Secure Communication:} The asymmetry of steering provides inherent directionality for secure quantum channels, particularly in untrusted or adversarial scenarios.

	\item \textbf{Quantum Internet and Cryptography:} Growing quantum internet frameworks (QIA, GSMA) identify steering-based protocols as key enablers for next-generation secure communications \cite{downlingQuantumTechnology2003}.

	\item \textbf{Quantum Key Distribution (QKD):} Device-independent QKD schemes benefit from steering's asymmetric nature, reducing hardware complexity.
\end{itemize}

\subsection{Steering vs Entanglement for Practical Applications}

Why focus on steering rather than entanglement?

\begin{itemize}
	\item \textbf{Resource Asymmetry:} Entanglement requires both parties to trust devices. Steering allows one untrusted party.

	\item \textbf{Operational Advantage:} Steering is a more practical resource for scenarios with asymmetric trust levels or capabilities.

	\item \textbf{Hierarchical Efficiency:} Since Steering $\subset$ Entanglement, steering-capable systems are automatically entanglement-capable, but the converse is not guaranteed.

	\item \textbf{Future-Oriented:} As quantum networks grow, asymmetric architectures (central trusted hubs, edge nodes) will dominate, making steering naturally suited for these topologies.
\end{itemize}

\subsection{Quantum Internet, Cryptography, and Beyond}

Emerging quantum communication infrastructure requires efficient, robust, and asymmetric quantum resources:

\begin{itemize}
	\item Quantum internet protocols prioritize asymmetric trust models
	\item Quantum cryptography standards (ETSI, ISO) increasingly focus on device-independent protocols
	\item Quantum sensing and metrology benefit from asymmetric correlations
	\item Quantum machine learning exploits steering for advanced protocols
\end{itemize}

% ============================= SECTION 2.8: CHAPTER SUMMARY =============================
\section{Chapter Summary}

This chapter has established the theoretical and motivational foundations for investigating quantum steering:

\begin{enumerate}
	\item \textbf{Quantum Correlations Hierarchy:} Discord $\supseteq$ Entanglement $\supseteq$ Steering $\supseteq$ Bell Nonlocality—steering occupies the optimal position for practical quantum information.

	\item \textbf{Continuous-Variable Advantage:} CV systems offer deterministic generation, optical compatibility, and Gaussian state formalism—ideal for steering studies.

	\item \textbf{Steering Fundamentals:} Asymmetric, directional, intermediate in strength, quantifiable for Gaussian states—uniquely suited for one-sided device-independent protocols.

	\item \textbf{Loss and Decoherence:} Prior work shows entanglement survives realistic loss in waveguides; steering robustness remains to be investigated.

	\item \textbf{Waveguides as Platforms:} Coupled waveguides demonstrate that passive systems can preserve entanglement. Active systems (like our RDQBL) may offer superior steering generation and control.

	\item \textbf{Practical Motivation:} Emerging quantum internet, cryptography, and sensing applications specifically require steering resources in asymmetric network topologies.
\end{enumerate}

The stage is now set for developing the specific RDQBL model (Chapter \ref{chp3}), investigating steering generation mechanisms (Chapter \ref{chp4}), and demonstrating practical feasibility (Chapter \ref{chp5}).

